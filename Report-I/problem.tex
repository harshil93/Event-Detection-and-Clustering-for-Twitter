\section{\uppercase{Problem Definition}}
Our overarching aim is to detect events using microblogs in social media (such as Twitter), track and model their evolution over time, and predict the occurrences of future events. Our focus will be on the following three sub-problems.
\subsection{Using LDA to detect events}
For the purpose of detection of events, we are focussing on the use of topic models, in particular Latent Dirichlet Allocation (LDA) \cite{blei2003latent}. Of late, LDA has been extensively exploited in literature in the context of topic modelling and text mining for finding topics from a corpus of text documents, assuming a bag-of-words model. Our aim is to use LDA on twitter data to cluster related keywords across millions of tweets under different \emph{topics}. Topics when associated with spatial and temporal data represent events. Specifically, our focus will be on the analysis of \emph{non-popular} events, deviating from the focus on popular events in recent literature. Event detection using Topic modeling techniques like LDA are unique in the sense that unlike analysis of bursty keywords, they not biased by event popularity. Hence, they can capture non-popular events as well.

\subsection{Evolution of events}
Once an event such as bomb blast, hurricane, and presidential speech have been identified through tweets, the next step is to track the evolution of these events over time and space. We are interested in investigating how they develop within their \emph{topic}, as well as analysing how their correlation to events in other topics changes over time. We aim to build an event evolution graph to study how one event triggers other events. Here also, our focus will be on analysing the activity and traffic pattern of the non-popular events. Another challenge is to adapting standard models employed for modeling the evolution of popular events \cite{lin2010pet} so that they can better model non-popular events in particular.

\subsection{Event prediction}
Finally, having modelled the events, our aim is to employ time-series models to predict future events. Specifically, we would like to focus on prediction of future occurrences of the same event, such as predict the onset of an annual endemic. Another interesting aspect is the prediction of outset of other events as a result of the event under consideration; for example, outset of protests followed by a child abuse case. A lot of work has already been done in this aspect. Our aim is to adapt existing techniques for more accurate predictions concerning non-popular events.